\begin{figure}[H]
    \includegraphics[width=\textwidth]{images/BackEnd.png}
    \centering
    \caption{Arquitectura del Backend de la Aplicación}
\end{figure}

\subsection{DAO}

El DAO (Data Access Object) será el encargado de administrar todos los accesos a la base de datos, permitiendo la interacción entre la aplicación y el almacenamiento de datos de manera estructurada y eficiente.

\subsection{API de Puntos de Venta}

Esta API gestionará las operaciones relacionadas con el punto de venta, permitiendo la generación de ventas, la administración de clientes y la gestión de los puntos de venta.

\subsection{API de Operaciones de Inventario}

Encargada de administrar el inventario de la tienda, permitiendo el control de stock, registro de movimientos de productos y actualización de información de los artículos disponibles.

\subsection{API de Administración de Inventario}

Facilita la gestión de productos y proveedores, asegurando el abastecimiento adecuado de la tienda y la correcta administración de los recursos.

\subsection{Manejador de Eventos}

El manejador de eventos procesará diversas acciones dentro del sistema, tales como ventas, asignaciones de tareas, eventos periódicos y recepciones de inventario. Su función principal es notificar a los módulos correspondientes, en especial al sistema de notificaciones.

\subsection{Sistema de Notificaciones}

El sistema de notificaciones será el encargado de enviar alertas y recordatorios a los destinatarios correspondientes según los eventos que lo activen. Dependiendo del tipo de evento, se generará una notificación específica.

\subsubsection{Notificaciones Periódicas}

Son aquellas notificaciones programadas por el administrador de la tienda para ser enviadas en intervalos regulares, conteniendo información relevante según la configuración establecida.

\subsubsection{Notificaciones Preventivas}

Notificaciones enviadas de manera anticipada en función de los eventos en curso dentro de la tienda, con el objetivo de prevenir posibles incidencias o tomar decisiones estratégicas a tiempo.

\subsubsection{Notificaciones de Alerta}

Se generan ante eventos extraordinarios o críticos que ocurran en la tienda. Estas alertas son enviadas de forma inmediata a los usuarios correspondientes para su pronta atención.

\subsection{Análisis de Datos}

Este módulo se encargará del procesamiento y análisis inteligente de la información disponible en el sistema, permitiendo obtener conocimientos útiles a partir de los datos almacenados.

\subsubsection{Pronósticos}

Proveerá predicciones sobre eventos futuros con base en datos históricos, buscando ofrecer resultados lo más precisos posible.

\subsubsection{Reportes}

Generará informes personalizados según los datos requeridos por el usuario, extrayendo y procesando la información relevante de la base de datos.

\subsection{Autenticación}

Este módulo verificará la identidad de los usuarios y validará sus permisos para acceder a funciones específicas dentro del sistema.

\subsubsection{Autenticación de Roles}

Validará las acciones que un usuario puede realizar dentro de una tienda específica, asegurando que solo accedan a las funcionalidades autorizadas.

\subsubsection{Autenticación de Tienda}

Se encargará de validar el acceso a una tienda determinada, asegurando que el usuario tenga los permisos necesarios para ingresar a la misma.
